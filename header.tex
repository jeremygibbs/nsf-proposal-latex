\documentclass{nsf_proposal}
%%%%%%%%%%%%%%%%%%%%%%%%%%%%%%%%%%%%%%%%%%%%%%%%%%%%%%%%%%%%%%%%%%%%%%%%
% LaTeX template to generate an NSF proposal
%
% First version by: Stefan Llewellyn Smith, Sarah Gille, others.
%
% Additions by: Ronni Grapenthin, New Mexico Tech.
%
% This version maintained/modified by: Jeremy A. Gibbs, Univ. of Utah
%
% This template is free source code. It comes without any warranty, to 
% the extent permitted by applicable law. You can redistribute it and/or 
% modify it under the terms of the Do What The Fuck You Want To Public 
% License, Version 2, as published by Sam Hocevar. See 
% http://www.wtfpl.net for more details.
%%%%%%%%%%%%%%%%%%%%%%%%%%%%%%%%%%%%%%%%%%%%%%%%%%%%%%%%%%%%%%%%%%%%%%%%
\usepackage{longtable}
\usepackage[latin1]{inputenc}
\usepackage[amssymb]{SIunits}
\usepackage{latexsym}
\usepackage{multirow}
\usepackage{wrapfig}
\usepackage{amsmath, amsthm, amssymb}
\usepackage{amsfonts}
\usepackage[format=plain,indention=0cm, font=small, labelfont=bf]{caption}
\usepackage{fancyhdr}
\usepackage[pdftex]{graphicx}
\usepackage[pdftex,
	    colorlinks, 	
	    pdfstartview=FitH,
	    linkcolor=black,
	    citecolor=black,
	    urlcolor=black,
	    filecolor=black
	    ]{hyperref}
\usepackage{lscape}
\usepackage[T1]{fontenc}
\usepackage{floatrow}
\usepackage{enumerate}
\usepackage{enumitem}
\usepackage{tabularx}
\usepackage{ragged2e}
\newcolumntype{Y}{ >{\RaggedRight\arraybackslash}X}
\newcommand\T{\rule{0pt}{2.6ex}}      
\newcommand\B{\rule[-1.2ex]{0pt}{0pt}}
\RequirePackage{color}
\definecolor{RED}{rgb}{1,0,0}
\definecolor{BLUE}{rgb}{0,0,1}
\definecolor{White}{rgb}{1,1,1}
\providecommand{\TODO}[1]{{\protect\color{red}\noindent {\bf [TODO]}\emph{#1} {\bf [/TODO]}}}
\providecommand{\todo}[1]{{\protect\color{red}\noindent {\bf [TODO]}\emph{#1} {\bf [/TODO]}}}
\providecommand{\CHECK}[1]{{\protect\color{blue} #1 (check) }}
\providecommand{\DummyText}[1]{{\protect\color{white} #1}}
\renewcommand{\refname}{References Cited}
\newcommand{\degrees}{$\!\!$\char23$\!$}
\def\rrr#1\\{\par
\medskip\hbox{\vbox{\parindent=2em\hsize=6.12in
\hangindent=4em\hangafter=1#1}}}
\def\baselinestretch{1}